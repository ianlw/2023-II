
\documentclass{article}
\usepackage{listings}
\usepackage{xcolor}
\usepackage{tcolorbox} % Asegúrate de que este paquete esté incluido

\definecolor{gray}{rgb}{0.5,0.5,0.5}
\definecolor{mygreen}{rgb}{0,0.6,0}
\definecolor{myorange}{rgb}{1,0.4,0}
\definecolor{myblue}{rgb}{0,0,0.8}

\lstdefinestyle{mystyle}{
    language=Python,
    basicstyle=\ttfamily,
    keywordstyle=\color{myblue},
    stringstyle=\color{myorange},
    commentstyle=\color{mygreen},
    numbers=none,
    numberstyle=\tiny\color{gray},
    breaklines=true,
    showstringspaces=false,
    frame=single,
    frameround=tttt, % Aquí se ajustan las esquinas redondeadas
    framesep=5pt,
}

\lstset{style=mystyle}

\begin{document}

\section{Código de Python en LaTeX con Esquinas Redondeadas}

\begin{lstlisting}
lista_puntos = []

def DDA(x0, y0, x1, y1):
    m = (y1 - y0) / (x1 - x0)
    y = y0
    for x in range(x0, x1 + 1):
        plot(x, int(y))
        lista_puntos.append((x, int(y)))
        y = y + m
\end{lstlisting}

\end{document}
