\documentclass{article}
\usepackage{amsmath}

\begin{document}


La transformada de Laplace de una función $f(t)$ se define como:

\[ L\{f(t)\} = \int_{0}^{\infty} e^{-st} f(t) \, dt \]

donde $s$ es una variable compleja, $t$ es la variable de la función $f(t)$, y $e^{-st} f(t)$ debe ser integrable en el intervalo $[0, \infty)$.\\

%%%%%%%%%%%%%%%%%%%%%%%%%

\[ L\{cos(t)\} = \int_{0}^{\infty} e^{-st} \cos(t) \, dt\]
\[ \int e^{-st} \cos(t) \, dt\]
\[ \int e^{-st} \sin(t) \, dt\]
La transformada de Laplace de \(t^n \cdot F(t)\) se expresa como:

\[ L\{t^n \cdot F(t)\} = (-1)^n \frac{d^n}{ds^n} L\{F(t)\} \]

%%%%%%%%%%%%%%%%%%%%%%%%%%%%%%%%%%%%%%%%%%

Para la integral $\int e^{-st} \cos(t) \, dt$, aplicamos integración por partes:

\begin{align*}
u &= e^{-st} & dv &= \cos(t) \, dt \\
du &= -se^{-st} \, dt & v &= \sin(t)
\end{align*}

La fórmula de integración por partes es:

\[
\int u \, dv = uv - \int v \, du
\]

Aplicando esto, obtenemos:

\begin{align*}
\int e^{-st} \cos(t) \, dt &= e^{-st} \sin(t) - \int \sin(t) \, (-se^{-st}) \, dt \\
\int e^{-st} \cos(t) \, dt &= e^{-st} \sin(t) + s \int \sin(t) \, e^{-st} \, dt \\
% &= e^{-st} \sin(t) + e^{-st} \cos(t) - \int e^{-st} \sin(t) \, dt
\end{align*}

Integraremos por segunda ves:

%%%%%%%%%%%%%%%%%%%%%%%%%%%%%%%%%%%%%%%
Para la integral $\int e^{-st} \sin(t) \, dt$, aplicamos integración por partes:

\begin{align*}
u &= e^{-st} & dv &= \sin(t) \, dt \\
du &= -se^{-st} \, dt & v &= -\cos(t)
\end{align*}


\begin{align*}
\int e^{-st} \sin(t) \, dt &= -e^{-st} \cos(t) - \int -\cos(t) \, (-se^{-st}) \, dt \\
\int e^{-st} \sin(t) \, dt &= -e^{-st} \cos(t) - s\int \cos(t) \, e^{-st} \, dt \\
\end{align*}

Reemplazamos

\begin{align*}
\int e^{-st} \cos(t) \, dt &= e^{-st} \sin(t) + s \int \sin(t) \, e^{-st} \, dt \\
\int e^{-st} \cos(t) \, dt &= e^{-st} \sin(t) + s(-e^{-st} \cos(t) - s \int \cos(t) \, e^{-st} \, dt) \\
\int e^{-st} \cos(t) \, dt &= e^{-st} \sin(t) + -se^{-st} \cos(t) - s^{2} \int \cos(t) \, e^{-st} \, dt) \\
(1+s^{2})\int e^{-st} \cos(t) \, dt &= e^{-st} \sin(t) + -se^{-st} \cos(t)\\
\int_{0}^{\infty} e^{-st} \cos(t) \, dt &= \frac{e^{-st} \sin(t) + -se^{-st} \cos(t)}{1+s^{2}}\\
% &= e^{-st} \sin(t) + e^{-st} \cos(t) - \int e^{-st} \sin(t) \, dt
\end{align*}

\begin{align*}
\int_{0}^{\infty} e^{-st} \cos(t) \, dt &= \frac{e^{-st} \sin(t) - se^{-st}\cos(t)}{1+s^{2}} \Bigg|_{0}^{\infty} \\
\int_{0}^{\infty} e^{-st} \cos(t) \, dt &= \frac{e^{-s(\infty)} \sin(\infty) + -se^{-s(\infty)} \cos(\infty)}{1+s^{2}} - \frac{e^{-s(0)} \sin(0) + -se^{-s(0)} \cos(0)}{1+s^{2}}\\
\end{align*}
exponencial elevado al -infinito es 0

\begin{align*}
% &= \frac{e^{-st} \sin(t) - se^{-st}\cos(t)}{1+s^{2}} \Bigg|_{0}^{\infty} \\
\int_{0}^{\infty} e^{-st} \cos(t) \, dt &= - \frac{-s}{1+s^{2}}\\
\int_{0}^{\infty} e^{-st} \cos(t) \, dt &= \frac{s}{1+s^{2}}\\
\end{align*}

\[ L\{cos(t)\} = \frac{s}{1+s^{2}}\]

\[ L\{t^{2}cos(t)\} = \frac{s}{1+s^{2}}\]

\[ L\{t^n \cdot F(t)\} = (-1)^n \frac{d^n}{ds^n} L\{F(t)\} \]
\[ L\{t^{2}cos(t)\} = (-1)^2 \frac{d^2}{ds^2} (\frac{s}{s^2+1}) \]
\[ L\{t^{2}cos(t)\} = \frac{d^2}{ds^2} (\frac{s}{s^2+1}) \]


\[ L\{t^{2}cos(t)\} = \frac{2s(s^2-3)}{(s^2+1)^3} \]

\begin{align*}
    \frac{d^2}{ds^2} (\frac{s}{s^2+1}) &= \frac{d}{ds}(\frac{s^2+1-s(2s)}{(s^2+1)^2})\\
    \frac{d^2}{ds^2} (\frac{s}{s^2+1}) &= \frac{d}{ds}(\frac{1-s^2}{(s^2+1)^2})\\
    \frac{d}{ds}(\frac{1-s^2}{(s^2+1)^2}) &= \frac{(-2s)(s^2+1)^2-(1-s^2)2(2s)(s^2+1)}{(s^2+1)^4}\\
    \frac{d}{ds}(\frac{1-s^2}{(s^2+1)^2}) &= \frac{2s^5-4s^3-6s}{(s^2+1)^4}\\
    \frac{d}{ds}(\frac{1-s^2}{(s^2+1)^2}) &= \frac{2s(s^2-3)(s^2+1)}{(s^2+1)^4}\\
    \frac{d}{ds}(\frac{1-s^2}{(s^2+1)^2}) &= \frac{2s(s^2-3)}{(s^2+1)^3}\\
\end{align*}


\end{document}


